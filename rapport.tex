\documentclass[french]{article}
\usepackage[T1]{fontenc}
\usepackage[utf8]{inputenc}
\usepackage{lmodern}
\usepackage[a4paper]{geometry}
\usepackage{babel}
\author{Badr YOUBI IDRISSI}
\title{Mini projet : évolution des tâches sur la matrice d'Heisenhower}
\begin{document}
\maketitle
\section{Introduction et Modélisation}

    La matrice d'Heisenhower est un outil d'organisation qui classe les tâches en fonction de leur 
    importance et leur urgence. Ce qui aide à se fixer des priorités. Le but de ce modèle est d'étudier
    l'évolution du travail restant à faire en fonction du temps, de l'urgence et de l'importance d'une tache.
    
    Pour pouvoir utiliser des EDP je me place dans un cadre continu : une tache est représenté par 
    une gaussienne de moyenne $(\mu_x,\mu_y)$ avec $\mu_x$ l'urgence de la tache et $\mu_y$ son importance.
    Ceci peut être justifié en pensant une tache comme plusieurs sous-taches d'importance et d'urgence variant
    autour de la moyenne. L'écart type $\sigma$ de la tache 

\end{document}
